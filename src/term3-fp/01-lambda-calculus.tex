\Subsection{Функциональная модель вычислений}
Типы прогрмаммирования:
\begin{enumerate}
    \item \textbf{Императивное} -- инструкции выполняются последовательно (общение с вычислителем). Результат -- выполнение последней инструкции.
    \item \textbf{Функциональное} -- программа это выражение, его выполнение это вычисление (редукция) выражения. Результат -- отсутствие редексов (подвыражения, которые могут быть вычислены непосредственно).
\end{enumerate}
\begin{definition}
    Связывание -- символ равенства ($a = 2 \cdot 7 + 1$), имя слева становится редексом (будет происходить подстановка наряду со встроенными правилами).
\end{definition}
\begin{example}
    $$
    z \cdot 4 + 1 \rightarrow (2 \cdot 7 + 1) \cdot 4 + 1 \rightarrow \dots
    $$
\end{example}

\begin{definition}
    Рекурсивное связывание -- символ равенства ($x = 2 \cdot x + 1$), имя слева становится редексом, такие выражения расходятся (так как нет терминирующего условия).
\end{definition}
\begin{example}
    $$
    x \cdot 2 \rightarrow (2 \cdot x + 1) \cdot 2 \rightarrow \dots
    $$
\end{example}

\begin{definition}
    Лямбда абстракция (анонимная функция) -- $\lambda \underbrace{y}_{\text{абстрактор}} \rightarrow \underbrace{2 \cdot y + 3}_{\text{тело}}$, чтобы применить функцию к аргументу, то мы записываем справа от тело аргумент.
\end{definition}

\begin{definition}
    Вычисление ($\beta$-редукция) -- просто подстановка вместо абстрактора самого аргумента.
\end{definition}

\begin{example}
    Заведем функцию:
    $$f = \lambda y \rightarrow 2 \cdot y + 1$$
\end{example}

Стретегии редукции:
\begin{enumerate}
    \item {
        В Haskell используется \textbf{ленивая} стратегия: сокращается самый левый внешний редекс: $(\lambda y \rightarrow 2 \cdot y + 3) (4 + 6) \rightarrow_{\beta} 2 \cdot (4 + 6) + 3 \rightarrow (8 + 12) + 3 \rightarrow 20 + 3 \rightarrow 23$
    }
    \item Была еще \textbf{энергетическая}, но я не успел. Кто-нибудь добавьте, если хочется.
\end{enumerate}

\begin{example}
    Тут был пример с факториалом.
\end{example}

\begin{definition}
    Функция нескольких переменных -- $\lambda n \rightarrow 2 \cdot m + 3 \cdot n$, тут свободная переменная это $m$, можно продолжить выражение, чтобы полуяиться замкунутое выражения (все переменные связанные): $\lambda m \rightarrow (\lambda n \rightarrow 2 \cdot m + 3 \cdot n)$.

    Вызываем функцию так: $(\lambda m \rightarrow (\lambda n \rightarrow 2 \cdot m + 3 \cdot n) 15) 4$ -- вместо $m$ подставится 15, а вместо $n$ -- 4.
\end{definition}


\Subsection{Чистое \lambda-исчисление}
\begin{definition}
    \lambda-терм -- переменная, либо апликация, либо абстракция. 
    $$x \in V \implies x \in \Lambda$$
    $$M, N \in \Lambda \implies (MN) \in \Lambda$$
    $$M \in \Lambda, x \in V \implies (\lambda x, M) \in \Lambda$$
\end{definition}
\begin{example}
    \lambda-термы:
    \begin{enumerate}
        \item x
        \item (x z)
        \item (\lambda x. (xz))
        \item ((\lambda x. (xz)) y)
        \item \dots
    \end{enumerate}
    Каждый следующий терм содержит предыдущий как подтерм.
\end{example}

\begin{remark}
    Имеются следующий обозначения:
    \begin{enumerate}
        \item  Внешние скобки опускаются
        \item Применение ассоциативно влево: $FXYZ == ((FX)Y)Z$
        \item Абстракция ассоциативна вправо: $\lambda x y z == \lambda x. (\lambda y. (\lambda z. M))$
    \end{enumerate}
\end{remark}

\begin{definition}
    \beta-редукция -- $(\lambda x. M) N \rightarrow_{\beta} [x \mapsto N] M$ -- подстановка $N$ вместо $x$ в $M$.
\end{definition}
\begin{definition}
    Применение вида $(\lambda x. M) N$, в которой левый аппликанд является абстракцией, называют \beta-редексом.
\end{definition}
\begin{definition}
    Шаг вычисления по приведенному выше правилу называют сокращением редекса.
\end{definition}
\begin{definition}
    В чистом \lambda-исчислении нет ничего кроме переменных, применения, абстракции и редукции.
\end{definition} 

todo
\begin{definition}
    Множество $FV(T)$ свободных переменных в терме $T$:
    $$FV(x) \implies \{x\}$$
    $$FV(M N) \implies FV(M) \cup FV(N)$$
    $$FV(\lambda x. M) = FV(M) \setminus \{x\}$$
\end{definition}

\begin{definition}
    Множество $BV(T)$ связных переменных в терме $T$:
    $$BV(x) \implies \emptyset$$
    $$BV(M N) \implies BV(M) \cup BV(N)$$
    $$BV(\lambda x. M) \implies BV(M) \cup \{x\}$$
\end{definition}


\begin{definition}
    М -- замкнутый \lambda-терм (комбинатор), если $FV(M) = \emptyset$. Множество замкнутых \lambda-термов обозначается $\Lambda^{0}$.
\end{definition} 
\begin{example}

    $I$ - комбинатор.
    $$I = \lambda x. x$$
    $$IM \rightarrow_{I} (\lambda x. x) M \rightarrow_{\beta} M$$
\end{example}
\begin{example}
    $$\omega = \lambda x. x x$$
    $$\Omega = \omega \omega \rightarrow_{\Omega} (\lambda x. x x) \omega \rightarrow_{\beta} \omega \omega$$
\end{example}

\begin{definition}
    \alpha-редукция -- $\lambda x. x \rightarrow_{\alpha} \lambda y. y$
\end{definition} 

\Subsection{Отношение эквивалентности на термах}
\begin{definition}
    Подстановка -- $[x \mapsto N]M$. Правила подстановки:
    \begin{enumerate}
        \item $[x \mapsto N]x = N$
        \item $[x \mapsto N]y = y$
        \item $[x \mapsto N](PQ) = ([x \mapsto N]P)([x \mapsto N]Q)$
        \item $[x \mapsto N](\lambda x. P) = \lambda x. P$
        \item $[x \mapsto N](\lambda y. P) = \lambda y. [x \mapsto N]P: y \notin FV(N)$ -- $y$ - не свободная переменная.
        \item $[x \mapsto N](\lambda y. P) = \lambda y'. [x \mapsto N]([y \mapsto y'] P): y \in FV(N)$ -- иначе.
    \end{enumerate}
\end{definition}

\begin{lemma}
    (О подстановке) Подстановки не коммутируют, однако верна $M, N, L \in \Lambda$. Предположим, что $x \not\equiv y$ и $x \notin FV(L)$. Тогда $[y \mapsto L]([x \mapsto N]M) \equiv [x \mapsto [y \mapsto L]N]([y \mapsto L]M)$
\end{lemma}
\begin{proof}
    Нудная индукция по всем 6ти случаям, с разбором всех подслучаев.
\end{proof}

\begin{definition}
    \beta-эквивалентность (хотим все свойства отношения эквивалентности). $\forall M, N \in \Lambda: (\lambda x. M) =_{\beta} [x \mapsto N]M$.
\end{definition}


Логические аксиомы этого правила:
\begin{enumerate}
    \item $M =_{\beta} M$ \\ 
    \item $M =_{\beta} N \implies N =_{\beta} M$ \\ 
    \item $M =_{\beta} N, N =_{\beta} L \implies M =_{\beta} L$ \\
\end{enumerate}


Правила совместимости:
\begin{enumerate}
    \item $M =_{\beta} M' \implies MZ =_{\beta} M'Z$ \\
    \item $M =_{\beta} M' \implies ZM =_{\beta} ZM'$ \\
    \item $M =_{\beta} M' \implies \lambda x. M =_{\beta} \lambda x. M'$ \\
\end{enumerate}

Если $M =_{\beta} N$ доказуемо в \lambda-исчислении, пишут $\lambda \vdash M =_{\beta} N$.

% todo:  табличка с аксиомами
\begin{definition}
    \alpha-эквивалентность -- $\lambda x. M =_{\alpha} \lambda y. [x \mapsto y] M$, если $y \notin FV(M)$ (переименование переменных). (там еще была табличка с аксиомами, посмотрите в презентации) 
\end{definition}

\begin{definition}
    \eta-эквивалентность -- $\lambda x. M x =_{\eta} M$. Смысл в том, что апликативное поведение термов слева и справа от знака равенства одинаоково: для произвольного $N$ верно: $(\lambda x. M x)N =_{\beta} MN$.
\end{definition}

