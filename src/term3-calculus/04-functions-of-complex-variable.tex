\Subsection{Голоморфные функции}

Если доказательство не указано, то оно повторяет то, что было в $\mathbb{R}$ (смотреть 1 семестр).

\begin{definition}
    $\Omega$ -- обсласть в $\mathbb{C}, \ f: \Omega \rightarrow \mathbb{C}, \ z_0 \in \Omega$.

    $f$ -- голоморфна в точке $z_0$, если существует $\lim_{z \rightarrow z_0} { \frac{f(z) - f(z_0)}{z - z_0} } =: f'(z_0)$.
\end{definition}
\begin{definition}
    $f$ комплексно дифф. в точке $z_0$, если $\exists k \in \mathbb{C}$:

    $f(z) = f(z_0) + k (z - z_0) + o(z - z_0)$ при $z \rightarrow z_0$.
\end{definition}

\begin{statement}
    $f$ -- голоморфна в точке $z_0 \Leftrightarrow f$ комплексно дифф. в точке $z_0$ и $k = f'(z_0)$.
\end{statement}

\begin{consequence}
    $f$ и $g$ голоморфны в точке $z_0$. Тогда 

    \begin{enumerate}
        \item {
            $f \pm g$ голом. в точке $z_0$
        }
        \item {
            $f \cdot g$ голом. в точке $z_0$
        }
        \item {
            Если $g(z_0 \not = 0)$, то $\frac{f}{g}$ голом. в точке $z_0$.
        }
        \item {
            Если $h$ голом. в точке $f(z_0)$, то $h \circ f$ голом. в точке $z_0$.
        }
    \end{enumerate}
\end{consequence}

\begin{remark}
    $f: \Omega \rightarrow \mathbb{C}$

    $z = x + iy, \ f(z) = f(x + iy) = g(x + i y) + i h(x + iy): \ g, h : \Omega \rightarrow \mathbb{R}$.

    $\frac{\partial f}{\partial x} (z_0) = \lim_{h \rightarrow 0, \ h \in \mathbb{R}} {\frac{f(z_0 + h) - f(z_0)}{h}} = f'(z_0)$.

    
    $\frac{\partial f}{\partial y} (z_0) = \lim_{h \rightarrow 0, \ h \in \mathbb{R}} {\frac{f(z_0 + i h) - f(z_0)}{h}} = \frac{f'(z_0)}{i} = -i \cdot f'(z_0)$.
\end{remark}

\begin{remark}
    $\binom{g(x + iy)}{h(x + iy)} = \binom{g(x_0 + i y_0)}{h(x_0 + iy_0)} + \binom{a \ b}{c \ d} \binom{a - x_0}{y - y_0} + o(|| (x - x_0, y - y_0) ||)$.

    $k = \alpha + i \beta$

    $k \cdot (z - z_0) = (\alpha + i \beta) ( (x - x_0) + i (y - y_0)) = \alpha(x - x_0) - \beta(y - y_0) + i (\beta (x - x_0) + \alpha (y - y_0))$

    Вещественная линейность + $\binom{\alpha \ -\beta}{\beta \ \alpha} \Leftrightarrow$ комплескная линейность.
\end{remark}

\begin{remark}
    Комплескная дифференцируемость $\Leftrightarrow$ вещественная дифференцируемость + матрица Якоби $\binom{\alpha \ \beta}{-\beta \ \alpha}$

    
    Комплескная дифференцируемость $\Leftrightarrow$ вещественная дифференцируемость + условия Коши-Римана $
    \begin{cases}
        \frac{\partial Re(f)}{\partial x} = \frac{\partial Im (f)}{\partial y} \\ 
        \frac{\partial Re(f)}{\partial y} = \frac{\partial Im (f)}{\partial x} \\ 
    \end{cases}
    $
\end{remark}
\begin{remark}
    $f(z) = f(z_0) + \underbrace{k}_{\in \mathbb{C}} (z - z_0) + o(z - z_0)$

    $k (z - z_0) = k w = |k| \cdot e^{i \phi} \cdot w, \ \phi = arg(k)$
\end{remark}


\begin{remark}
    Обозначения.

    $\frac{\partial}{\partial z} = \frac{1}{2} \cdot \left( \frac{\partial}{\partial x} - i \frac{\partial}{\partial y} \right)$
    
    $\frac{\partial}{\partial \overline{z}} = \frac{1}{2} \cdot \left( \frac{\partial}{\partial x} - i \frac{\partial}{\partial y} \right)$

    $dz = dx + i dy$

    $d \overline{z} = dx - i dy$

    $df = \frac{\partial f}{\partial x} \cdot dx + \frac{\partial f}{\partial y} dy = \frac{\partial f}{\partial z} dz + \frac{\partial f}{\partial \overline{z}} d \overline{z}$
\end{remark}

\begin{theorem}
    \textbf{Условия Коши-Римана}.

    $f: \Omega \rightarrow \mathbb{C}, \ a \in \Omega$

    $f$ -- дифф. в точке $a$ как функция из $\mathbb{R}^2$ в $\mathbb{R}^2$. Следующие условия равносильны:

    \begin{enumerate}
        \item {
            $f$ -- голоморфна в точке $a$.
        }
        \item {
            $d_a f$ -- комплексно линеен
        }
        \item {
            условия Коши-Римана
        }
        \item {
            $\frac{\partial f}{\partial \overline{z}} (a) = 0$
        }
    \end{enumerate}
\end{theorem}

\begin{proof}
    Мы выяснили все, кроме $(3) \Leftrightarrow (4)$:

    $\frac{\partial f}{\partial \overline{z}} = 0 \Leftrightarrow \frac{\partial f}{\partial x} + i \frac{\partial f}{\partial y} = 0 \Leftrightarrow \frac{\partial (Re(f) + i Im(f))}{\partial x} + i \cdot \frac{\partial (Re (f) + i Im (f))}{\partial y} = 0 \Leftrightarrow \begin{cases}
        \frac{\partial Re(f)}{\partial x} - \frac{\partial Im (f)}{\partial y} = 0 \\ 
        \frac{\partial Im(f)}{\partial x} + \frac{\partial Re(f)}{\partial y} = 0
    \end{cases}$ -- а это и есть условия Коши-Римана.
\end{proof}

\begin{remark}
    Обозначения.

    $f \in H(\Omega) \Leftrightarrow f : \Omega \rightarrow \mathbb{C}$ и голоморфна во всех точках из $\Omega$.
\end{remark}
\begin{consequence}
    $\Omega$ -- область, $f \in H(\Omega)$ и $Im(f) = const \implies f = const$
\end{consequence}
\begin{proof}
    $\frac{\partial Im (f)}{\partial y} = 0 \implies \frac{\partial Re(f)}{\partial x} = 0$

    $\frac{\partial Im (f)}{\partial x} = 0 \implies \frac{\partial Re(f)}{\partial y} = 0$

    $\implies Re(f) = const$
\end{proof}
\begin{theorem}
    \textbf{Коши} (ah, shit, here we go again...)

    $f \in H(\Omega) \implies f(z) dz$ локально точная.
\end{theorem}
\begin{proof}
    Будет два разных док-ва.

    \begin{enumerate}
        \item {
            Для случая непрерывно-дифф. $\frac{\partial Re(f)}{\partial x}, \dots$ (имеются в виду все частные производные).

            Тогда замкнутость $\implies$ локальная точность.

            $f(z) dz = f(z) (dx + i dy) = (Re(f) + i \cdot Im(f)) \cdot (dx + i dy) = Re(f) dx - Im(f) dy + i (Im (f) dx + Re(f) dy)$.

            $P dx + Q dy$ -- замкн. $\Leftrightarrow \frac{\partial P}{\partial y} = \frac{\partial Q}{\partial x}$

            $Re(f) dx - Im(f) dy$ -- замкн. $\Leftrightarrow \frac{\partial Re(f)}{\partial y} = - \frac{\partial Im(f)}{\partial x}$

            $Im(f) dx + Re(f) dy$ -- замкн. $\Leftrightarrow \frac{\partial Im(f)}{\partial y} = \frac{\partial Re(f)}{\partial x}$
        }
        \item {
            Общий случай.

            % todo: picture
            Надо доказать, что интеграл по любому прямоугольнику (со сторонами параллельными осям) из круга (круг берем вокруг точки из $\Omega$. Добавьте картинку, плиз) равен 0.

            От противного: пусть нашелся прямоугольник $P$, т.ч. $\alpha (P) := \int_P {f(z) dz } \not = 0$.

            % todo: picture
            Режем прямоугольник на 4 части, индексируем как $P^{1}, P^{2}, P^{3}, P^{4}$, строим объоды каждого (против часовой стрелки). Тогда $\alpha(P) = \alpha(P^{1}) + \alpha(P^{2}) + \alpha(P^{3}) + \alpha(P^{4})$, $|\alpha(P)| \leq |\alpha(P^{1})| + |\alpha(P^{2})| + |\alpha(P^{3})| + \alpha(P^{4})$.

            Хотя бы одно из слагаемых $\geq \frac{1}{4} |\alpha(P)|$, назовем такое $P_1$ (индекс уже снизу!). Разрежем его на 4 равные части. Пусть $P_2$ такой, что $|\alpha(P_2)| \geq \frac{1}{4} |\alpha (P_1)|$ и т.д.

            $|\alpha(P_n)| \geq \frac{1}{4^n} |\alpha (P)|$.

            % todo: picture
            Берем $a$ из $P_n$:

            $f(z) = f(a) + f'(a) (z - a) + o(z - a)$

            $\alpha (P_n) = \int_{P_n} { f(z) dz } = \underbrace{\int_{P_n} { f(a) dz }}_{= 0} + \underbrace{\int_{P_n} { f'(a) (z - a) dz }}_{= 0} + \int_{P_n} { o(z - a) dz }$

            $o(z - a) = (z - a) \cdot \beta (z - a)$, где $\beta(z - a) \underbrace{\rightarrow}_{z \rightarrow a} 0$

            $\left| \int_{P_n} {(z - a) \beta (z - a) dz} \right| \leq max_{z \in P_n} { |z - a| \cdot |\beta (z - a)| } \cdot \underbrace{l(P_n)}_{\text{периметр}} \leq max_{z \in P_n} { |\beta (z - a)| } \cdot \frac{l(P)}{2^n} \cdot \frac{c}{2^n} \implies$

            $\implies \frac{|\alpha (P)|}{4^n} \leq |\alpha(P_n)| \leq \frac{l(P) \cdot c}{4^n} \cdot max_{z \in P_n} |\beta (z - a)| \implies max_{z \in P_n} |\beta (z - a)| \geq \frac{|\alpha (P)|}{l(P) \cdot c} > 0$ -- противоречие.
        }
    \end{enumerate}
\end{proof}
\begin{consequence}
    \begin{enumerate}
        \item {
            Если $f \in H(\Omega)$, то у каждой точки $a \in \Omega$ есть окрестность, в которой существует ф-я $F$, т.ч. $F' = f$ в этой окрестности.

            \begin{proof}
                Пусть $F$ первообразная формы $f(z) dz$. Поймем, что $F' = f$.

                $\frac{\partial F}{\partial x} = f(z), \ \frac{\partial F}{\partial y} = i \cdot f(z) \implies \frac{\partial F}{\partial x} + i \frac{\partial F}{\partial y} = 0 \implies \frac{\partial F}{\partial \overline{z}} = 0$
            \end{proof}
        }
        \item {
            $f \in H(\Omega)$, $\gamma$ стягиваемый в $\Omega$ путь $\implies \int_{\gamma} { f(z) dz } = 0$
        }
    \end{enumerate}
\end{consequence}
\begin{theorem}
    $f \in C(\Omega), \ \Delta$ -- прямая параллельная оси координат.

    $f \in H(\Omega \setminus \Delta)$

    Тогда $f(z) dz$ локально точная.
\end{theorem}
\begin{proof}
    Надо проверять, что интеграл по довольно маленькому прямоугольнику (со стороронами паралл. осям) это 0.

    % todo: picture!!!
    Очевидно, что если прямоугольник не пересекает $\Delta$, то там все очевидно. Хотим рассматривать только те, что задевают. Те, что пересекают $\Delta$, можно разбить на две части (верхнюю и нижнюю). По каждой из частей будет 0, тогда и в сумме тоже будет 0. То есть нас вообще интересуют только те прямоугольники, у которых $\Delta$ это одна из сторон. Рассмотрим их:
    
    % todo: picture

    Тут мастхэв картинка, на которой мы откусывает подпрямугольник размера $\epsilon$.

    $\int_{P_{\epsilon}} { f(z) d z } = 0 \rightarrow_{\epsilon \rightarrow 0} \int_{P} { f(z) dz }$

    $\left|\int_{P} {f(z) dz} - \int_{P_{\epsilon}} { f(z) dz } \right| \leq |\int_{1} + \int_{3}| + |\int_{2}| + |\int_{4}|$

    $\left| \int_{2} {f(z) dz} \right| \leq M \cdot (\text{длина } 2) = M \epsilon$

    $\left| \int_{1} + \int_{3} \right| = \left| \int_{a}^{b} { \left(f (x + i y_0) - f(x + i(y_0 + \epsilon)) \right) dx } \right| \leq \int_{a}^{b} { |\dots| dx } = (*)$

    $f$ непрер. на компакте $\implies$ равномерно непрер.

    $\forall \gamma > 0: \ \exists \epsilon > 0$ если $\rho (\text{аргумент}) < \epsilon \implies |f(\dots) - f(\dots)| < \gamma$, тогда 

    $(*) \leq (b - a) \cdot \gamma$
\end{proof}

\begin{consequence}
    $f: \Omega \rightarrow \mathbb{C}$

    $f \in C(\Omega)$ и $f$ голоморфна в $\Omega$ за исключением мн-ва изолированных точек, тогда форма $f(z) dz$ все равно лок. точная.
\end{consequence}
\begin{proof}
    Рассмотрим окр-ть, в которой ровно одна плохая точка.

    % todo: picture
    Давайте проведем прямую через это точку, тогда работает теорема.
\end{proof}

\begin{definition}
    Индекс кривой отн-но точки $Ind(\gamma, z_0)$.

    $\gamma$ -- замкнутая кривая, не проходящая через точку $z_0$.

    $Ind(\gamma, 0) = \frac{\phi(b) - \phi(a)}{2\pi} \in \mathbb{Z}$ -- кол-во оборотов $\gamma$ вокруг 0.

    $\gamma: [a, b] \rightarrow \mathbb{C}$

    $\gamma(t) = r(t) e^{i \phi(t)}$, $\phi$ -- непрерывна (полярная замена).
\end{definition}